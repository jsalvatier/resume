\documentclass[11pt,a4paper]{moderncv}
\moderncvtheme[red]{classic}
%\usepackage{enumitem}
\usepackage[utf8]{inputenc}
\usepackage[scale=0.8]{geometry}
\recomputelengths
\usepackage{enumitem}

\AfterPreamble{
\definecolor{linkcolour}{rgb}{0,0.2,0.6}
\hypersetup{colorlinks,breaklinks,urlcolor=linkcolour, linkcolor=linkcolour}
}

%pretty C++
\newcommand{\CPP}
{C\nolinebreak[4]\hspace{-.05em}\raisebox{.22ex}{\footnotesize ++}}

%pretty C#
\newcommand{\CS}
{C\nolinebreak[4]\hspace{-.05em}\raisebox{.22ex}{\footnotesize\#}}

\newcommand{\bitem}{\begin{samepage}\begin{itemize} \itemsep -2pt}
\newcommand{\eitem}{\end{itemize}\end{samepage} }

\newcommand{\entry}[6]
{\cventry{#1}{#2}{#3}{#4}{#5}{
    \bitem 
    #6
    \eitem
    }
}

\newcommand{\itemc}[5]
{\cventry{#4}{#1}{#2}{#3}{}{#5}
}


%----------------------------------------------------------------------------------
%            content
%----------------------------------------------------------------------------------

% personal data
\firstname{John}
\familyname{Salvatier}
%\address{1900 NE 68th St}{98115 Seattle, WA}
\phone{360-602-1069}
\email{jsalvatier@gmail.com}


\begin{document}
\maketitle
\section{Relevant Experience}
    \entry{2009 -- 10/2012}{Developer/Quant}{RPX Research, Inc.}{Redmond, WA}{}{
        \item Engineered on-line, high-frequency, predictive price model for a bond trading algorithm (\CS)
        \item Engineered time-series models to look for profitable bond, futures and equity market trading strategies (\CS, Python)
        \item Engineered system for generating and tracking trading performance metrics (\CS)
        \item Built system for evaluating bond, futures and equity trading strategies against historical market data (\CS) 
        \item Added high-frequency price collection to data collection infrastructure (\CS)
        \item Built and improved-existing automated trading infrastructure (\CS)
    }

    \entry{Summer 2006}{Developer (Intern)}{Capstone Technology}{Camas, WA}{}{
        \item Improved stability and interface efficiency of PARCSuite plant operations management software (\CS)
        \item Responsible for the migration of several components of the PARCSuite software from the 1.1 .NET framework to the 2.0 .NET framework 
    }
\closesection{}

\section{Open Source}
    \cventry{}{Github account}{}{\href{https://github.com/jsalvatier}{github.com/jsalvatier}}{}{}

    \cventry{2012 -- }{\href{https://github.com/pymc-devs/pymc/commits/pymc3\#readme}{PyMC 3.0}}{Bayesian inference package (Python)}{}{}{}
        \cvline{}{\href{https://github.com/pymc-devs/pymc/commits/pymc3}{
        Engineered PyMC 2.2 replacement with dramatically simpler, smaller and more powerful codebase, which will soon replace PyMC 2.2 and become PyMC 3.0}}

    \cventry{2010 -- 2012}{\href{https://github.com/pymc-devs/pymc\#readme}{PyMC 2.0}}{Bayesian inference package (Python)}{}{}{}
        \cvline{}{Overhauled likelihood calculation to automatically provide gradients }
        \cvline{}{\href{https://github.com/jsalvatier/gradient\_samplers/blob/master/gradient\_samplers}{Implemented gradient based samplers which scale better with problem size, self-tune, handle difficult distributions well, etc.}}
        \cvline{}{\href{http://pypi.python.org/pypi/multichain\_mcmc}{Engineered PyMC extension allowing for multiple chain samplers}}
        \cvline{}{\href{https://github.com/pymc-devs/pymc/commits/numexpr\_dist}{Experimented with numexpr and Cython code generation for likelihoods}}

    \cventry{2009}{\href{packages.python.org/scikits.bvp\_solver}{scikits.bvp\_solver}}{}{}{}{
        Built and presently maintain a user-friendly interface for the Fortran numerical \\
        boundary value problem solver BVP\_SOLVER
        }

\closesection{}

\section{Self-Study}
    \cventry{10 -- 12/2012}{Carnegie Mellon Courses}{}{}{}{
        Completed all lectures and homework for two courses. Courses were designed for Standard ML, but I completed them in Scala.}
        \cvline{}{\href{http://www.cs.cmu.edu/~15150/previous-semesters/2012-spring/}{15-150: Functional Programming}}
        \cvline{}{\href{http://www.cs.cmu.edu/~15210/index.html}{15-210: Parallel \& Sequential Data Structures and Algorithms}}
    \cventry{12/2012 -- }{Hadoop}{}{}{}{
        Learning Hadoop via Twitter's Scalding, by implementing parallel \\
        Scan function for Scalding
    }

\closesection{}

\section{Skills}
    \subsection{Languages}
        \cvline{Fluent}{\CS, Python, Scala}
        \cvline{Experienced}{Java, Haskell, Standard ML, C, \CPP, R, \LaTeX\ and others}

    \subsection{Other}
        \cvline{}{Experienced with Bayesian statistical modeling (Markov Chain Monte-carlo)}
        \cvline{}{Well-versed in economics and decision theory}
        \cvline{}{Skilled at technical writting}
        \cvline{}{Fluent in Spanish}

\closesection{}
\section{Education} 
    \cventry{2009}{University of Washington}{B.S. in Chemical Engineering}{}{}{}
    \cventry{2009}{University of Washington}{B.S. in Paper Science and Engineering}{}{}{}
    
\closesection{}
\section{Other Experience}
    \cventry{Summer 2008}{Process Engineer (Intern)}{Boise-Cascade}{Pasco, WA}{}{}
        \cvline{}{Investigated economics and feasibility of three capital projects}
        \cvline{}{Conducted trial to investigate systemic product quality measurement problems}

    \cventry{Summer 2007}{Process Engineer (Intern)}{Boise-Cascade}{Pasco, WA}{}{}
        \cvline{}{Investigated maintenance and energy projects for cost effectiveness }

    \cventry{Summer 2005}{Researcher (Intern)}{Kimberly-Clark}{Neenah, WI}{}{}
\closesection{}
\end{document}
