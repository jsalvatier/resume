% LaTeX file for resume 
% This file uses the resume document class (res.cls)

\documentclass[margin]{res}
% the margin option causes section titles to appear to the left of body text 
\textwidth=5.2in % increase textwidth to get smaller right margin
%\usepackage{helvetica} % uses helvetica postscript font (download helvetica.sty)
%\usepackage{newcent}   % uses new century schoolbook postscript font 
\usepackage{xcolor,hyperref}

\newcommand{\bitem}{\begin{samepage}\begin{itemize} \itemsep -2pt}
\newcommand{\eitem}{\end{itemize}\end{samepage} }


\newcommand{\hrowbase}[3]{
  {\bf #1} #2 \hfill #3
}

\newcommand{\headrow}[3]{
  \hrowbase{#1}{#2}{#3}
    \vspace{6pt}
    \bitem
      }
\newcommand{\eheadrow}[0]{\eitem}


\newcommand{\headrowdesc}[3]{
  \hrowbase{#1}{}{#2} \ \\
    #3
    \vspace{6pt}
      }

\newcommand{\email}[1]{ \href{mailto:#1}{#1}}

\definecolor{marineblue2}{rgb}{0.05,0.1,0.5}
\colorlet{linkcol}{marineblue2}
\hypersetup{colorlinks,breaklinks,linkcolor=linkcol,urlcolor=linkcol,anchorcolor=linkcol,citecolor=linkcol}

%pretty C++
\newcommand{\CPP}
{C\nolinebreak[4]\hspace{-.05em}\raisebox{.22ex}{\footnotesize\bf ++}}

%pretty C#
\newcommand{\CS}
{C\nolinebreak[4]\hspace{-.05em}\raisebox{.22ex}{\footnotesize\#}}

\begin{document} 

\resumewidth=6.5in
\newsectionwidth{1.0in}
\name{\LARGE John Salvatier}
\begin{resume} 
 

\section{Relevant Experience}
  \headrow{Developer/Quant}{RPX Research, Inc., Redmond, WA}{6/2009--10/2012}
     \item Engineered on-line, high-frequency, predictive price model for a bond trading algorithm (\CS)
     \item Engineered time-series models to look for profitable bond, futures and equity market trading strategies (\CS, Python) 
     \item Engineered system for generating and tracking trading performance metrics 
     \item Built system for evaluating bond, futures and equity trading strategies against historical market data  
     \item Added high-frequency price collection to data collection infrastructure 
     \item Built and improved-existing automated trading infrastructure 
  \eheadrow

  \headrow{Developer (Intern)}{Capstone Technology, Camas, WA}{Summer 2006}
    \item Improved stability and interface efficiency of PARCSuite plant operations management software (\CS)
    \item Responsible for the migration of several components of the PARCSuite software from the 1.1 .NET framework to the 2.0 .NET framework 
  \eheadrow

 \section{Open Source}
 {\bf Github account} \href{https://github.com/jsalvatier}{github.com/jsalvatier}

    \headrow{\href{https://github.com/pymc-devs/pymc/commits/pymc3\#readme}{PyMC 3.0}}{Bayesian inference package (Python)}{2012--Present}
      \item \href{https://github.com/pymc-devs/pymc/commits/pymc3}{Engineered PyMC 2.2 replacement with dramatically simpler, smaller and more powerful codebase, which will soon replace PyMC 2.2 and become PyMC 3.0}
    \eheadrow

    \headrow{\href{https://github.com/pymc-devs/pymc\#readme}{PyMC 2.0}}{Bayesian inference package (Python)}{2010--2012}
      \item Overhauled likelihood calculation to automatically provide gradients 
      \item \href{https://github.com/jsalvatier/gradient\_samplers/blob/master/gradient\_samplers}{Implemented gradient based samplers which scale better with problem size, self-tune, handle difficult distributions well, etc.}
      \item \href{http://pypi.python.org/pypi/multichain\_mcmc}{Engineered PyMC extension allowing for multiple chain samplers}
      \item \href{https://github.com/pymc-devs/pymc/commits/numexpr\_dist}{Experimented with numexpr and Cython code generation for likelihoods}
    \eheadrow

    \headrowdesc{\href{packages.python.org/scikits.bvp\_solver}{scikits.bvp\_solver}}{2009}{
      Built and presently maintain a user-friendly interface for the Fortran numerical \\
    boundary value problem solver BVP\_SOLVER}

\section{Self-Study}
    \headrowdesc{Carnegie Mellon Courses}{10/2012--12/2012}{Completed all lectures and homework for two courses. Courses were \\
    designed for ML, but I completed them in Scala.} 
    \bitem
\item \href{http://www.cs.cmu.edu/~15150/previous-semesters/2012-spring/}{15-150: Functional Programming}
      \item \href{http://www.cs.cmu.edu/~15210/index.html}{15-210: Parallel \& Sequential Data Structures and Algorithms}
    \eitem
    \headrowdesc{Hadoop}{12/2012--Present}{Learning Hadoop via Twitter's Scalding, by implementing parallel \\
      Scan function for Scalding}

\section{Skills}
    \bitem
      \item Fluent in \CS, Python, Scala
      \item Experience with Java, Haskell, ML, C, \CPP, R, \LaTeX\ and others
      \item Experienced with Bayesian statistical modeling (Markov Chain Monte-carlo)
      \item Well-versed in economics and decision theory
      \item Skilled at technical writing
      \item Fluent in Spanish 
    \eitem

\section{Education} 
  \headrow{University of Washington}{}{2009}
      \item B.S. in Chemical Engineering
      \item B.S. in Paper Science and Engineering
  \eheadrow
    
\section{Other \\ Experience}
    \headrow{Process Engineer (Intern)}{Boise-Cascade, Pasco, WA}{Summer 2008}
      \item Investigated economics and feasibility of three capital projects 
      \item Conducted trial to investigate systemic product quality measurement problems 
    \eheadrow

    \headrow{Process Engineer (Intern)}{Boise-Cascade, Pasco, WA}{Summer 2007}
      \item Investigated maintenance and energy projects for cost effectiveness 
    \eheadrow

    \hrowbase{Researcher (Intern)}{Kimberly-Clark, Neenah, WI}{Summer 2005 }

\section{Contact}
     Phone: 360-602-1069 \\
     Email: \email{jsalvatier@gmail.com}


\end{resume} 
\end{document} 

