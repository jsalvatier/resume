\documentclass[11pt,a4paper]{moderncv}
\moderncvtheme[red]{classic}
\usepackage{enumitem}
\usepackage[utf8]{inputenc}
\usepackage[margin=.85in, scale=0.8]{geometry}
\recomputelengths
\usepackage{enumitem}

\AfterPreamble{
\definecolor{linkcolour}{rgb}{0,0.2,0.6}
\hypersetup{colorlinks,breaklinks,urlcolor=linkcolour, linkcolor=linkcolour}
}

%pretty C++
\newcommand{\CPP}
{C\nolinebreak[4]\hspace{-.05em}\raisebox{.22ex}{\footnotesize ++}}

%pretty C#
\newcommand{\CS}
{C\nolinebreak[4]\hspace{-.05em}\raisebox{.22ex}{\footnotesize\#}}

\newcommand{\bitem}{\begin{samepage}\begin{itemize}[leftmargin=*, labelindent=8pt, itemsep=-1pt]}
\newcommand{\eitem}{\end{itemize}\end{samepage}}

\newcommand{\entry}[6]
{\cventry{#1}{#2}{#3}{#4}{#5}{
    \bitem 
    #6
    \eitem
    }
}

\newcommand{\itemc}[5]
{\cventry{#4}{#1}{#2}{#3}{}{#5}
}



%----------------------------------------------------------------------------------
%            content
%----------------------------------------------------------------------------------

% personal data
\firstname{John}
\familyname{Salvatier}
%\address{1900 NE 68th St}{98115 Seattle, WA}
\phone{360-602-1069}
\email{jsalvatier@gmail.com}
\homepage{github.com/jsalvatier}


\begin{document}
\maketitle
\vspace*{-5mm}

\section{Open Source}
    \cventry{2012 -- current}{\href{https://github.com/pymc-devs/pymc/commits/pymc3\#readme}{PyMC 3.0}}{Bayesian inference package (Python)}{}{}{
        \item Lead author of PyMC 2.2 Theano based replacement with dramatically simpler, smaller and more powerful code-base, which will soon replace PyMC 2.2 and become PyMC 3.0}
        \item Engineered transparent missing value imputation 
        \item Engineered automatic transformations for constrained variables to be unconstrained for more efficient training
        \item Lead author on Probabilistic Programming in Python using PyMC (2016, forthcoming)
        \item \href{https://www.youtube.com/watch?v=VVbJ4jEoOfU}{Presented on PyMC3 at PyDataConf Seattle 2015}
    }

    \cventry{Jan 2016}{\href{https://github.com/jsalvatier/deep-go/}{Deep-Go}}{Replicating Google's Go Neural Net (Lua/Torch)}{}{}{
        \href{https://github.com/jsalvatier/deep-go/}{
        Replicating Move Evaluation in Go Using Deep Convolutional Neural Networks (2014) and exploring different models and training strategies.
        \bibitem{notes} Madison (2014)  {\em Move Evaluation in Go Using Deep Convolutional Neural Networks} arXiv.org, http://arxiv.org/abs/1412.6564.
    }

    \entry{2010 -- 2012}{\href{https://github.com/pymc-devs/pymc\#readme}{PyMC 2.0}}{Bayesian inference package (Python, C, Fortran)}{}{}{
        \item Added Automatic Differentiation for likelihoods
        \item \href{https://github.com/jsalvatier/gradient\_samplers/blob/master/gradient\_samplers}{Implemented gradient based samplers which scale better with problem size, self-tune, handle difficult distributions well, etc.}
        \item \href{http://pypi.python.org/pypi/multichain\_mcmc}{Engineered PyMC extension allowing for multiple chain samplers}
        \item \href{https://github.com/pymc-devs/pymc/commits/numexpr\_dist}{Experimented with numexpr and Cython code generation using Jinja2 templating for likelihoods}
    }
    \cventry{2012}{NumPy}{(C)}{}{}{
        \href{https://github.com/numpy/numpy/pull/377}{Patch adding advanced indexing interface to NumPy's C-API}
    }
    \cventry{2012}{Theano}{(Python, C)}{}{}{
        \href{https://github.com/Theano/Theano/pull/1083}{Patch adding fast advanced indexing and gradient support}
    }

    \cventry{2009}{\href{packages.python.org/scikits.bvp\_solver}{scikits.bvp\_solver}}{(Python, Fortran)}{}{}{
        Built and presently maintain a user-friendly interface for the Fortran numerical \\
        boundary value problem solver BVP\_SOLVER
        }

\section{Experience}
    \entry{10/2015 -- current}{Researcher}{AI Impacts}{Berkeley, CA}{}{
        \item Design and planning work for AI Progress survey
        \item AI researcher interviews investigating AI progress forecasts and AI milestones
        \item Sociology of AI research 
    }
    \entry{4/2013 -- 7/2015}{Software Development Engineer II}{Amazon.com, Inc.}{Seattle, WA}{}{
        \item Taught functional data programming for scala and javascript to two teams
        \item Became internally recognized expert in the Spark mapreduce framework
        \item Critical part of project to rebuild Contribution Profit system in Spark
        \item Conceived of Retail Video Recorder for recording fast help videos for business users and then lead prototype team
    }
    \entry{2009 -- 10/2012}{Developer/Quantitative Analyst}{RPX Research, Inc.}{Redmond, WA}{}{
        \item Engineered on-line, high-frequency (10ms), price model for a bond trading algorithm with continuous updating, extendible model and data-feed specific tuning (\CS)
        \item Engineered price, trade and volatility time-series models for large datasets in search of bond, futures and equity market trading strategies using SQL, NumPy and PyMC (\CS, Python)
        \item Engineered system for creating and visualizing trading performance metrics using SQL and MSChart and with higher-order function based extensions  
        \item Built system for evaluating bond, futures and equity trading strategies against historical market data by simulating real time trading 
        \item Added high-frequency price data to data collection infrastructure 
    }
    \cventry{2012}{Statistical Consulting}{}{}{}{
        Corporate bond default model
        \vspace*{-4pt}
        \bitem
        \item Bayesian proportional-hazards model with a latent, time-varying, financial-fragility factor 
        \item Multi-level effects
        \item Fit numerically using two-layer Hamiltonian Monte-Carlo
            \eitem
    }
\closesection{}

    \begin{thebibliography}{1}
        \bibitem{notes} Salvatier J., Wiecki T., Fonnesbeck C. (2016)  {\em Probabilistic Programming in Python using PyMC} PeerJ Computer Science, forthcoming.
    \end{thebibliography}


\closesection{}

\section{Technical}
    \vspace{-24pt}
\cvitem{}{
    \begin{samepage}\begin{itemize}[leftmargin=*, itemsep=-2pt]
    \item Expert with \CS, Python, Scala, PyMC3, Spark, Pandas
    \item Experienced with Java, Haskell, Standard ML, C, \CPP, Fortran, R, \LaTeX\
    \item Fluent with Bayesian statistical modeling and sophisticated Monte-Carlo sampling
    \item Well-versed in economics and decision theory
    \item Skilled at technical writing
        \eitem
}

\closesection{}
\section{Education} 
    \cventry{2009}{University of Washington}{B.S. in Chemical Engineering}{}{}{}
    \cventry{2009}{University of Washington}{B.S. in Paper Science and Engineering}{}{}{}
    
\closesection{}
\end{document}
